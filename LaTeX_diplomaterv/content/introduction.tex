%----------------------------------------------------------------------------
\chapter{Bevezetés}
%----------------------------------------------------------------------------

Manapság egyre több technológia jelenik meg, aminek alapja az NLP (Natural Language Processing). Ezek a technológiák nagy része nem lenne megvalósítható szemantikai elemzés nélkül. A szemantikai elemzés célja, hogy egy nyers szövegből, vagy beszédhangból előállítsa annak a szemantikai reprezentációját. Ez a reprezentáció egy irányított gráf is lehet, amit ha a mondat szintaktikai szerkezetét reprezentáló fákból állítunk elő, akkor a teljes feladat felfogható egy gráf-transzformációként. 

Bár szemantikai elemzésre számos Deep Learning-es megoldás létezik, ezek pontatlansága nagy igényt teremt egy analitikus mély szemantikai elemzési módszerre. A gráf transzformációs megközelítés ígéretes eredményeket mutatott fel, mint például a Stanford Parser, ami TREGEX-ek segítségével végzi el tiszta analitikus módon a transzformációkat.
Több formalizmus is létezik a transzformációk leírására, mint például a HRG (hyperedge-replacement grammar) vagy az IRTG(interpreted regular tree grammars). Jelenleg is egy ezekkel kapcsolatos kutatás folyik az AUT tanszéken.

A kutatás során az ALTO(Algebraic Language Toolkit)-val dolgoztunk, ami a jelenlegi leghatékonyabb környezet IRTG-k futtatására. Ugyanakkor a kutatásnak állandó gátját jelenti, hogy az IRTG még egy fejletlen nyelvtan és nehezen átlátható; és az ALTO-ból is hiányoznak fontos funkcionalitások. A problémán sokat enyhítene, ha az IRTG szabályokat REGEX-ek segítségével is meglehetne hivatkozni.

Szakdolgozatom keretében egy templatelésre alkalmas nyelvet fejlesztettem ki, ami a Slime fantázianévre hallgat. Segítségével az IRTG nyelvtanokat tömörebben és átláthatóbban lehet definiálni. Mivel az ALTO java-ban készül, a nyelv Kotlinban készül ANTLRv4 segítségével. Még nincs teljesen kifejlődve, de a feladathoz szükséges megoldásokat tartalmazza. Ilyen például a template definiálás, egymásba ágyazás, regexxel hivatkozás és sok egyéb. Teljes formájában egy univerzális bővítmény lesz, ami bármely nyelv vagy szöveg felett használható.
 	 	 	
A dolgozat a következőképpen épül fel: Az 1. fejezetben bemutatom a …., a 2. fejezetben a ….-ról írok, majd a 3. fejezetben a …., végül ….
